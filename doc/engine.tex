\documentclass{book}
\usepackage{listings}
\usepackage{xcolor}
\usepackage{textcomp}
\usepackage[T1]{fontenc}

\lstset{
  showstringspaces=false,
  upquote=true,
  language=c++,
  frame=single, % draw a frame at the top and bottom of the code block
  tabsize=4, % tab space width
  basicstyle=\ttfamily,
  keywordstyle=\color{blue}\ttfamily,
  stringstyle=\color{red}\ttfamily,
  commentstyle=\color{olive}\ttfamily,
  morecomment=[l][\color{magenta}]{\#},
  numbers=left, % display line numbers on the left
  columns=fullflexible,
  frame=single,
  breaklines=true,
  postbreak=\mbox{\textcolor{red}{$\hookrightarrow$}\space},
}

\begin{document}
\chapter{Scripting API}
\section{ANSI Color}
All strings displayed to the user may be colorized by use of special, in-string
color codes, which the engine will translate into ANSI escape sequences. ANSI
24-bit color, supported by most modern clients and terminal emulators, may also
be used with hexadecimal RGB sequences (as in HTML).

\begin{figure}[h!]
  \caption{Color codes}
  \begin{center}
    \begin{tabular}{|c|c|}
      \hline
      \texttt{\char0 black\char0} &Black foreground\\
      \texttt{\char0 blue\char0} &Blue foreground\\
      \texttt{\char0 cyan\char0} &Cyan foreground\\
      \texttt{\char0 default\char0} & reset terminal (bg/fg)\\
      \texttt{\char0 green\char0} &Green foreground\\
      \texttt{\char0 magenta\char0} &Magenta foreground\\
      \texttt{\char0 white\char0} &White foreground\\
      \texttt{\char0 yellow\char0}&Yellow foreground\\

      \texttt{\char0 bgblack\char0}&Black background\\
      \texttt{\char0 bgblue\char0}&Blue background\\
      \texttt{\char0 bgcyan\char0}&Cyan background\\
      \texttt{\char0 bggreen\char0}&Green background\\
      \texttt{\char0 bgmagenta\char0}&Magenta background\\
      \texttt{\char0 bgred\char0}&Red background\\
      \texttt{\char0 bgwhite\char0}&White background\\
      \texttt{\char0 bgyellow\char0}&Yellow background\\

      \texttt{\char0 blink\char0}&Blink (default resets)\\
      \texttt{\char0 rvid\char0}&Reverse video (default resets)\\

      \texttt{\char0 \#RRGGBB\char0}& Foreground to hex RGB color\\
      \texttt{\char0 @RRGGBB\char0}&Background to hex RGB color\\
      \hline
    \end{tabular}
  \end{center}
\end{figure}

For example, the command \texttt{player.Send("\char0 red\char0 WARNING\char0
  default\char0");} would send the player red text that says WARNING. (Note that
it is good practice to end any colorized string with \texttt{\char0
  default\char0} to keep the last color used from bleeding into the next text.)


\section{Global Event Handlers}

There are several global events for which the scripter must define handler
functions. Low level tasks like I/O, connection management, and command
scheduling are done by the engine and do not need to be dealt with by the
scripter in these.

\subsection{\texttt{void GameTick()}}
Called by the engine every second. Use to perform global game-specific logic
which must occur periodically and synchronously with respect to itself.

\subsection{\texttt{void OnPlayerConnect(PlayerConnection@ conn)}}
\texttt{OnPlayerConnect()} is called by the engine each time a player connection
is made. A \texttt{PlayerConnection} object representing the new connection is
passed by the engine as an argument to this event handler, and can be used to
receive input and other connection-related events from that player.

\texttt{PlayerConnection} has two functions which set the callbacks for the
input and disconnect events of its particular player. The callbacks may be any
functions, but an obvious configuration is for them to be member functions
of a Player class, as in the following example:

\begin{lstlisting}
  array<Player@> g_players;

  void OnPlayerConnect(PlayerConnection@ conn)
  {
    try
    {
      Player@ hPlayer = Player(conn);
      g_players.insertLast(hPlayer)
    }
    catch
    {
      Log("Exception thrown!\r\n" + getExceptionInfo() + "\r\n");
    }

  }

  class Player
  {
    weakref<PlayerConnection> m_connection;
    Player(PlayerConnection@ conn)
    {
      @m_connection = conn;
      conn.SetInputCallback(InputCallback(OnInputReceived));
      conn.SetDisconnectCallback(DisconnectCallback(OnDisconnect));
    }

    void OnInputReceived(string input)
    {
      // Handle player input here
    }

    void OnDisconnect()
    {
      // Handle player disconnect here
    }

    // Other Player related stuff
  };

\end{lstlisting}


\section{Global Functions}
\begin{itemize}
\item \texttt{void Log(string msg)}\\
  Prints a message to the server log.
\item \texttt{void HashPassword(const string@ input, string@ output)}\\
  Hashes string \texttt{input} and places the result into \texttt{output}.
\item \texttt{void TrimString(const string@ input, string@ output)}\\
  Trims string \texttt{input} of whitespace from the left and right, and places
  the result into \texttt{output}.
\end{itemize}

\section{Global Objects}
Presently the only global object is \texttt{game\_server}. It may be removed
later and its member functions replaced with global ones which behave like
static functions.

\subsection{Server game\_server}
\begin{itemize}
\item \texttt{game\_server.SendToAll(string msg)}\\
  Sends a message to all connected players.

\item \texttt{game\_server.QueueScriptCommand(ICommand@ cmd, uint32 delay)}\\
  Queues command cmd for execution delay seconds in the future.
\end{itemize}

\section{Interfaces and Built-in Classes}
\subsection{Multiprecision numbers}
The engine supports multiprecision numbers through two built-in types: \texttt{MPInt}
and \texttt{MPFloat}, which are multiprecision integers and floating point numbers,
respectively. The values they represent can be almost arbitrarily large and precise.

Both types support basic arithmetic, and will soon be accompanied by their own
versions of the usual mathematical functions.

Multiprecision numbers should not be used unless they are absolutely required,
as they cost many times more overhead in performance and memory than their
hardware-supported counterparts.

\subsection{uuid}

\texttt{uuid} is a class built into the engine which stores and generates
version 4 UUIDs.  Internally, the data are stored as 128 bit integers, though
they can be converted to and from their standard string representations with the
\texttt{ToString} and \texttt{FromString} member functions.

UUIDs are ideal for use as object identifiers - because of the strong guarantee
of uniqueness, the same ID can be used both as the object's in-memory identifier
and as its database key.

\begin{itemize}
\item \texttt{void FromString(string uuid\_as\_string)}\\
  Converts a UUID from provided string representation and stores it in this object.

\item \texttt{string ToString()}\\
  Returns string representation of stored UUID

\item \texttt{void Generate()}\\
  Generates a new, random UUID in this object.

\end{itemize}

\subsection{ICommand}

Commands may be executed immediately or queued for execution at a later time,
and may represent player commands, character actions, or game events. They may
be placed into per-player command queues or the global event queue.

Commands must implement the ICommand interface and its method \texttt{int
  opCall()}, which is called when the command is run; arguments to the command
may be passed through its constructor parameters and stored in the object until
execution.

\begin{itemize}
\item \texttt{void Send(string msg)}\\ Sends the player a message. As with all
  displayed text, the message may be colorized through use of special color
  codes enclosed in backticks.

\item \texttt{void Disconnect()}\\
  Forcibly disconnects the player.

\item \texttt{void QueueCommand(ICommand@ cmd, uint32 delay\_in\_seconds, uint32
  delay\_in\_nanoseconds)}\\ Enqueues an ICommand to run delay\_in\_seconds +
  delay\_in\_nanoseconds in the future.

\end{itemize}

\chapter{Database API}

Classes with objects to be persisted must implement the \texttt{IPersistent}
interface, which defines methods for storage, retrieval, and schema
initialization. Classes which implement \texttt{IPersistent} are given special
treatment by the engine, which calls its event handling methods and invisibly
manages low-level database access.

The database is relational. Columns can be added to tables freely or
ignored, which makes the database easily forwards and backwards compatible.

Every instantiable class is given its own table in the database, composed of
columns corresponding to its properties and those of its parent classes, if
they, too, implement \texttt{IPersistent}.

Database tables, columns, and rows are how object data are stored.
\begin{itemize}
\item A table corresponds to a class
\item The table's columns correspond to the properties (member variables) of the class
\item Rows correspond to the objects created belonging to the class, each with all the properties of
  its class but with different values for them
\end{itemize}

\section{IPersistent}

\textbf{The superclass's \texttt{OnDefineSchema}, \texttt{OnSave}, and
  \texttt{OnLoad} member functions should be called in the subclass
  implementations before anything else is done in them}.


\subsection{\texttt{void OnDefineSchema(DBTable@ table)}}
When the engine starts the scripts, it creates an empty table object for every
\texttt{IPersistent} class, then calls each class's \texttt{OnDefineSchema}
method and gives it the table object. The class table object can then be
customized inside \texttt{OnDefineSchema} by calling \texttt{DBTable}'s methods
to add columns to the table schema.


The actual saving and loading of an object is done by the \texttt{DBSaveObject}
and \texttt{DBLoadObject} family of global functions, which take as their
primary arguments the key of the object to store or retrieve. Keys may be
integers, text strings, or UUIDs. The mapping between the object in memory
properties and in database columns is done by the \texttt{OnSave} and
\texttt{OnLoad} event functions, which are called by the engine when
\texttt{DBSaveObject} and \texttt{DBLoadObject} are invoked by the scripts; the
event functions should not themselves be called directly, as the engine must
perform internal bookkeeping for class hierarchies and subtables (which are
specialized DBTables designed to persist arrays of aggregate data belonging to a
containing object).



\section{DBTable}
\texttt{void AddUUIDCol(string column\_name, KeyType keytype, DBTable@ foreign\_table)}
  Adds a UUID column to the table. The KeyType argument can be omitted or set to
\texttt{DBKEYTYPE\_PRIMARY}, if this column should be the table's primary key. The \texttt{foreign\_table}
  argument should be omitted unless the key type is set to \texttt{DBKEYTYPE\_FOREIGN}.
\texttt{void AddTextCol(string column\_name, KeyType keytype, DBTable@ foreign\_table)}
  Same as above, but for holding text data.
\texttt{void AddRealCol(string column\_name, KeyType keytype, DBTable@ foreign\_table)}
  Same as above, but for holding single or double precision floating point.
\texttt{void AddIntCol(string column\_name, KeyType keytype, DBTable@ foreign\_table)}
  Same as above, but for holding 8, 16, 32, and 64 bit integers.
\texttt{void AddMPIntCol(string column\_name, KeyType keytype, DBTable@ foreign\_table)}
  Adds a column to store an multiprecision integer.
\texttt{void AddMPFloatCol(string column\_name, KeyType keytype, DBTable@ foreign\_table)}
  Adds a column to store a multiprecision float.

\end{document}
